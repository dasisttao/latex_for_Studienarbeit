\section*{Kurzfassung}
Diese Arbeit basiert haupts�chlich auf der vorhandenen Fahrzeugarchitektur von IfF, um die Einrichtung und Implementierung des statischen Umgebungsmodells um das Auto in \ac{ROS} durchzuf�hren. Basierend auf ROS und C++ Programmierung konzentriert sich die Implementierung des Modells auf die Realisierung des inversen Sensorsystems und des Binary-Bayes-Filters. Zus�tzlich werden einige Features bzw. Funktionen hinzugef�gt, wie die Ausgabe des Arrays, in dem die A-posteriori-Wahrscheinlichkeit jeder Gitterzelle gespeichert ist, und die Visualisierung von Bewegungspfaden f�r die nachfolgende Systemerweiterung. Am Ende dieser Arbeit werden die Evaluation des Umfeldmodells und die M�glichkeit seiner Verbesserung und Erweiterung er�rtert.