\section{Einleitung}
\label{sec:Einleitung}
Im ersten Kapitel wird auf die Einf�hrung des Themas, das zu erreichende Ziel und der Aufbau dieser Arbeit eingegangen.

\subsection{Einf�hrung}
Das Thema \glqq Autonomes Fahren\grqq{} ist in den vergangen Jahren immer weiter in den Vordergrund ger�ckt. Selbstfahrende Autos verlagerten sich allm�hlich von Laborentwicklungs- und Testbedingungen auf �ffentliche Stra�en.
Die Konzept,dass Autos autonom fahren k�nnen, besch�ftigte die Menschen schon in den 1920er-1930er-Jahren.
\\Um das Konzept praxistauglich zu machen, braucht es das kombinierte Wissen unter anderem aus Informatik, Fahrzeugtechnik, Sensorik, Mechatronik.
Unter dem Begriff \glqq Autonomes Fahren\grqq{} werden Fahrzeuge gefasst, die ohne Eingriff und �berwachung durch einen Menschen selbstst�ndig sowie ziel gerichtet im Stra�enverkehr fahren k�nnen.
\\Ingenieure f�r selbstfahrende Autos verfolgen aktuell zwei wesentliche und unterschiedliche Herangehensweisen f�r autonome Entwicklung, welche aus Robotik-Ansatz und Deep-Learning-Ansatz bestehen. In den vergangenen zwei Jahrzehnten hat der Robotik-Ansatz mit einer Menge von Beitr�gen der zahlreichen Wissenschaftler und Technik sowohl in Akademie als auch in technologisch f�hrenden Unternehmen gro�en Fortschritt gemacht.
\\Im Rahmen dieser Arbeit handelt es sich um ein statisches Umfeld. Sollte die Echtzeitanforderung ber�cksichtigt, ist der Deep-Learning Ansatz zu verzichten, denn es ist mit vielen Literaturen bewiesen, dass ein statisches Umfeld mit Robotik-Ansatz zutreffend und effizient modelliert werden kann.\cite{Thrun.2005}
\\Die zentrale Idee des Robotik-Ansatzes ist, dass die Unsicherheit in der Robotik sich unter Verwendung der Wahrscheinlichkeitsrechnung darstellen l�sst.
\subsection{Zielstellung}
\label{sec2:Zielstellung}
Um das automatisierte Fahren zu verwirklichen, wird die Umfelderfassung bzw. die Umfeldmodellierung des Fahrzeugs zugrunde gelegt. Das Ziel der vorliegende Arbeit ist auf Modellierung und dazu ihre Implementierung gerichtet.  

\subsection{Aufbau}
Nach dieser Einleitung wird im Kapitel 2 die technische Grundlagen, welche sich eng mit Erfassung und Modellierung eines statischen Umfelds beziehen.
\\ 




