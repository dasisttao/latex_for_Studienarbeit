\section{Einleitung}
\label{sec:Einleitung}
Im ersten Kapitel wird auf die Einf�hrung des Themas, das zu erreichende Ziel und den Aufbau dieser Arbeit eingegangen.

\subsection{Einf�hrung}
In den letzten Jahren hat das Thema \glqq Autonomes Fahren\grqq{} zunehmend an Bedeutung gewonnen. Selbstfahrende Autos verlagern sich allm�hlich von Laborentwicklung und Testbedingungen auf �ffentliche Stra�en. Die Konzept, dass Autos autonom fahren k�nnen, besch�ftigt die Menschen schon seit den 1920er-1930er-Jahren.
\\Um diese Konzept umzusetzen, m�ssen Kenntnisse aus Informationstechnologie, Fahrzeugtechnik, Sensortechnologie und Mechatronik kombiniert werden. Der Begriff \glqq Autonomes Fahren\grqq{} kann als unabh�ngiges und zielgerichtetes Fahren eines Fahrzeugs im Stra�enverkehr ohne menschliches Eingreifen oder Aufsicht verstanden werden. Ingenieure verwenden aktuell zwei wesentliche und unterschiedliche Methoden der autonomen Fahrzeugentwicklung, einschlie�lich Robotermethode und Deep-Learning-Methode. In den vergangenen zwei Jahrzehnten hat der Robotik-Ansatz mit einer Menge von Beitr�gen zahlreicher Wissenschaftler und Techniker sowohl in Akademie als auch in technologisch f�hrenden Unternehmen gro�e Fortschritte gemacht.
\\In dieser Arbeit liegt der Schwerpunkt auf der Modellierung eines statischen Umfeldmodell. Sollte die Echtzeitanforderung ber�cksichtigt werden, ist hierbei die Deep-Learning-Methode zu verzichten. Zudem zeigt eine gro�e Menge an Literatur, dass statische Umgebungsmodelle durch Robotermethoden zutreffend und effizient erstellt werden k�nnen~\cite{Thrun.2005}. Die zentrale Idee Robotik-Ansatzes ist, dass die Unsicherheit der Robotik oder des autonomen Fahrens durch eine Wahrscheinlichkeitsmethode dargestellt wird.
\subsection{Zielstellung}
\label{sec2:Zielstellung}
Um autonomes Fahren zu realisieren, wird die Umfelderfassung bzw. die Umfeldmodellierung des Fahrzeugs zugrunde gelegt. Ziel dieser Arbeit ist es, die relevante theoretische Grundlage durch Recherche zu erhalten und auf dieser Grundlage die Erstellung und Implementierung des Umfeldmodells in \acf{ROS} durchzuf�hren. Es handelt sich bei dieser Arbeit um statische Umgebungen in st�dtischen R�umen.
\subsection{Aufbau}
Nach dieser Einleitung werden in Kapitel 2 die technische Grundlagen vorgestellt, die in engem Zusammenhang mit der Erfassung und Modellierung statischer Umgebungen stehen. Hierbei wird haupts�chlich die Entwicklung von Umfeldmodellen und die Erl�uterung von Sensoren samt Sensormodellen vorgestellt, die zur Erkennung der Umgebung erforderlich sind.
\\ Kapitel 3 ist der Kern dieser Arbeit und erl�utert die tats�chliche Implementierung des Umfeldmodells in Kombination mit~\ac{ROS}. Zus�tzlich werden einige Features bzw. Funktionen vorgestellt, die dem Modell hinzugef�gt wurden, um den nachfolgenden Datenaustausch mit anderen Modulen und die Erweiterung des Modells zu erleichtern.
\\Nach der Implementierung konzentriert sich Kapitel 4 auf die Evaluation des Modells. Dies dient haupts�chlich dazu, die Qualit�t der erstellten Karte und die Berechnungsgeschwindigkeit des Modells zu bewerten.
\\Anschlie�end wird in Kapitel 5 die M�glichkeiten der Verbesserung und der Erweiterung des Modells er�rtert und in Kapitel 6 schlie�t die Arbeit mit einer Zusammenfassung. 





