\section{Zusammenfassung}
F�r die Realisierung des autonomen Fahrens im urbanen Raum wurde das Ziel dieser Arbeit festgelegt, ein statisches Umfeldsmodell zu erstellen, das die Umgebung des Fahrzeugs darstellen kann. Die bestehende \ac{IfF}-Fahrzeugarchitektur, einschlie�lich Versuchsfahrzeuge und Ibeo-Lux-Laserscanner, wurde verwendet, um die Umgebung zu erfassen und das entsprechende Umfeldmodell zu entwickeln.
\\Mit diesem Ziel wird in dieser Arbeit nach Recherchen eine gitterbasierte Karte unter Verwendung der Wahrscheinlichkeitstheorie verwendet. Diese Karte wird als \og{} bezeichnet und in zwei Ebenen unterteilt. In der ersten Ebene wird der Raum in einzelne Zellen diskretisiert und jede Zelle kann als besetzt oder frei betrachtet werden. Hierbei wird eine vereinfachte Annahme getroffen, dass die Belegungszust�nde zwischen den Zellen unabh�ngig voneinander sind und sich nicht gegenseitig beeinflussen. In der zweiten Ebene wird die Wahrscheinlichkeitsmethode eingef�hrt, um diesen Belegungszustand zu beschreiben. Der Kern hier ist die Verwendung von \bb{}, um die Belegungswahrscheinlichkeit jeder Zelle abzusch�tzen. Der \bb{}r ist eine rekursive Berechnungsmethode, um die Wahrscheinlichkeit eines Zustands zu erhalten. Er erfordert eine Echtzeit-Eingabe von Sensordaten und ein Sensormodell, mit dem die Messdaten in Wahrscheinlichkeiten der Zellen abgebildet werden. Daher wird in dieser Arbeit anhand der technischen Daten von Ibeo-LUX-Laserscanner das inverse Sensormodell erl�utert. Dieses Modell umfasst Hinderniskartierung, Freiraummodellierung und Beschreibung unbekannter Gebiete. Seine Verwendung wandelt die Sensordaten in die aktuelle Sch�tzung der Belegungswahrscheinlichkeit jeder Gitterzelle um und kombiniert mit Hilfe von \bb{} die historisch akkumulierte Belegungswahrscheinlichkeit, um die aktuelle neue Wahrscheinlichkeitssch�tzung zu erhalten. Nat�rlich wird diese Sch�tzung auch als nachfolgende historische Belegungswahrscheinlichkeit dienen. Die Auswahl bzw. die Erstellung dieses Sensormodells ist einer der gr��ten Faktoren, die die Performanz und Genauigkeit des gesamten Umfeldmodells beeinflussen. Daher ist es auch ein Einstiegspunkt f�r die Verbesserung des Modells in zuk�nftigen Forschungsarbeiten.
\\Ein weiterer Schwerpunkt dieser Arbeit ist die Implementierung des Modells in \ac{ROS}. Hier wurden der Arbeitsmechanismus und die Eigenschaften von \ac{ROS} vorgestellt, insbesondere die Funktionen in Bezug auf Modellierung. Nach Vergleich und Diskussion wird das \emph{Gridcells}-Display zur Visualisierung �bernommen. Durch seine Verwendung wird die Visualisierung der Karte flexibler, was f�r die sp�tere Systemerweiterung und das Debuggen praktisch ist. Einige der verwendeten kleinen Details und m�glichen Probleme sind in dieser Arbeit ebenfalls speziell aufgef�hrt. Dar�ber hinaus ist die Beziehung und Umrechnung zwischen dem Weltkoordinatensystem, dem Fahrzeugkoordinatensystem und dem Ankerkoordinatensystem eines der Schl�sselelemente, um die korrekte Implementierung des Modells sicherzustellen, das ebenfalls in dieser Arbeit ausf�hrlich vorgestellt wird. Lasersensordaten werden als Eingabe an das System �bertragen, und die Punktwolke wird gem�� der Datenqualit�t und den Modellanforderungen, wie z. B. der Kartengr��e, gefiltert. Dar�ber hinaus m�ssen die Daten �ber den entsprechenden Algorithmus in \ac{ACS} konvertiert werden, da der Anker des Modells der Ausgangspunkt f�r die Raumdiskretisierung des Umfeldmodells ist. Es ist anzumerken, dass die �bertragungsfrequenz von Lasersensordaten nicht mit der Frequenz anderer Daten wie GPS-Daten �bereinstimmt. Aus diesem Grund muss die �bertragungsfrequenz aus verschiedenen Datenquellen im Modell synchronisiert werden. Die Implementierung des inversen Sensormodells verwendet Raycasting mit Bresenham-Algorithmus, um den Zellen innerhalb des erfassbaren Bereichs Belegungswahrscheinlichkeiten zuzuweisen. Diese Werte werden in einem zweidimensionalen Array \emph{current\_grid} gespeichert, sodass sie an jeder Position in Zellen indiziert werden k�nnen. Analog dazu wird die A-posteriori-Wahrscheinlichkeit nach dem Bianry-Bayes-Filter in einem anderen zweidimensionalen Array \emph{history\_grid} gespeichert, um als g�ltige Daten an andere Systeme ausgegeben und zm n�chsten Zeitpunkt als die A-priori-Wahrscheinlichkeit verwendet zu werden. In dieser Arbeit wird auch erw�hnt, dass das vom vorherigen Zeitpunkt erhaltene \emph{history\_grid} aktualisiert werden muss, wenn sich die Position des Ankerpunkts �ndert. Auf der Grundlage des Umfeldmodells werden f�r die nachfolgenden Anforderungen der Systemerweiterung und -erweiterung einige Features bzw. Funktionen hinzugef�gt, wie z. B. die Auswahl des Ausgangsinformationsfluss, die Visualisierung von Ego-Fahrzeug und Bewegungspfaden. 
\\Nach Abschluss der Implementierung wird die Evaluation des Umfeldmodells anhand von Qualit�t der Karte und Rechenleistung durchgef�hrt. Die Arbeit endet mit der Er�rterung der M�glichkeiten der Modellverbesserung und -erweiterung und dieser Zusammenfassung.

